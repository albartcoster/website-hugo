\documentclass[]{article}
\usepackage{lmodern}
\usepackage{amssymb,amsmath}
\usepackage{ifxetex,ifluatex}
\usepackage{fixltx2e} % provides \textsubscript
\ifnum 0\ifxetex 1\fi\ifluatex 1\fi=0 % if pdftex
  \usepackage[T1]{fontenc}
  \usepackage[utf8]{inputenc}
\else % if luatex or xelatex
  \ifxetex
    \usepackage{mathspec}
  \else
    \usepackage{fontspec}
  \fi
  \defaultfontfeatures{Ligatures=TeX,Scale=MatchLowercase}
\fi
% use upquote if available, for straight quotes in verbatim environments
\IfFileExists{upquote.sty}{\usepackage{upquote}}{}
% use microtype if available
\IfFileExists{microtype.sty}{%
\usepackage{microtype}
\UseMicrotypeSet[protrusion]{basicmath} % disable protrusion for tt fonts
}{}
\usepackage[margin=1in]{geometry}
\usepackage{hyperref}
\hypersetup{unicode=true,
            pdftitle={Verdroogde mais},
            pdfauthor={Albart Coster},
            pdfborder={0 0 0},
            breaklinks=true}
\urlstyle{same}  % don't use monospace font for urls
\usepackage{graphicx,grffile}
\makeatletter
\def\maxwidth{\ifdim\Gin@nat@width>\linewidth\linewidth\else\Gin@nat@width\fi}
\def\maxheight{\ifdim\Gin@nat@height>\textheight\textheight\else\Gin@nat@height\fi}
\makeatother
% Scale images if necessary, so that they will not overflow the page
% margins by default, and it is still possible to overwrite the defaults
% using explicit options in \includegraphics[width, height, ...]{}
\setkeys{Gin}{width=\maxwidth,height=\maxheight,keepaspectratio}
\IfFileExists{parskip.sty}{%
\usepackage{parskip}
}{% else
\setlength{\parindent}{0pt}
\setlength{\parskip}{6pt plus 2pt minus 1pt}
}
\setlength{\emergencystretch}{3em}  % prevent overfull lines
\providecommand{\tightlist}{%
  \setlength{\itemsep}{0pt}\setlength{\parskip}{0pt}}
\setcounter{secnumdepth}{0}
% Redefines (sub)paragraphs to behave more like sections
\ifx\paragraph\undefined\else
\let\oldparagraph\paragraph
\renewcommand{\paragraph}[1]{\oldparagraph{#1}\mbox{}}
\fi
\ifx\subparagraph\undefined\else
\let\oldsubparagraph\subparagraph
\renewcommand{\subparagraph}[1]{\oldsubparagraph{#1}\mbox{}}
\fi

%%% Use protect on footnotes to avoid problems with footnotes in titles
\let\rmarkdownfootnote\footnote%
\def\footnote{\protect\rmarkdownfootnote}

%%% Change title format to be more compact
\usepackage{titling}

% Create subtitle command for use in maketitle
\newcommand{\subtitle}[1]{
  \posttitle{
    \begin{center}\large#1\end{center}
    }
}

\setlength{\droptitle}{-2em}

  \title{Verdroogde mais}
    \pretitle{\vspace{\droptitle}\centering\huge}
  \posttitle{\par}
    \author{Albart Coster}
    \preauthor{\centering\large\emph}
  \postauthor{\par}
      \predate{\centering\large\emph}
  \postdate{\par}
    \date{2018-08-14}


\begin{document}
\maketitle

Na weken bijzonder droog weer weten we dat een deel van de mais
verdroogd is. Een belangrijke vraag is wanneer deze mais gehakseld dient
te worden, en hoe.

In het algemeen geldt dat we eerst zeker moeten zijn dat de mais echt
verdroogd is en er geen kans op enig herstel of verdere groei van het
gewas is voor de beslissing tot hakselen wordt genomen. Als het gewas
nog bij kan trekken en gedeeltelijk verder kan groeien moet het niet
gehakseld worden. Als de kans op verdere groei echter afwezig is is
hakselen de beste keus.

Het is verstandig om een verloren gewas te hakselen omdat schadelijke
microben zoals gisten en schimmels in aantallen toenemen in het gewas
als het dood op het land staat. Verder is een dood of bijna dood gewas
zwakker en zal het makkelijker gaan liggen met sterke wind of regen,
waardoor de verliezen nog groter zijn.

Over het algemeen kunnen we verwachten dat een gewas met weinig zetmeel
veel vocht bevat. Dit komt omdat het drogestof percentage van de korrel
bij de maisoogst meestal tussen de 50 en 60\% ligt. Bij snijmais met
35\% zetmeel kunnen we berekenen dat het korrelaandeel meer dan 50\% van
de drogestof van de mais bedraagt. Het is dus logisch dat snijmais met
minder zetmeel over het algemeen natter is dan snijmais met meer
zetmeel. Herhaalde ervaringen met verdroogde snijmais in Spanje
bevestigen dit beeld.

Omdat het gewas door de aanhoudende droogte verzwakt is kunnen we
verwachten dat er veel schadelijke microben in aanwezig zijn. Het
inkuilproces is een soort strijd tussen goede, pH verlagende bacterien,
en slechte bacterien. Doel van het inkuilproces is de groei van de goede
bacterien, zoals \emph{Lactobacillus Plantarum}, harder in aantal
toenemen dan schadelijke microorganismen zoals \emph{Clostridia},
\emph{Listeria}, gisten, en schimmels. Omdat de schadelijke soorten
gevoelig zijn voor een lage pH is het voor een juist verloop van het
inkuilproces belangrijk dat de pH zo snel mogelijk daalt. Dit wordt
bevorderd door een juist inkuilproces (schoon werken, goed verdichten
zodat er weinig zuurstof in de kuil overblijft, snel afdekken, en
gebruik van juiste inkuilmiddelen, en een juiste afsluiting van de
kuil).

Voor een snelle daling van de pH raden wij aan om inkuilmiddelen
gebaseerd op melkzuurvormende melkzuurbacterien te gebruiken. Let
hierbij op de aantallen melkzuurbacterien (meer is over het algemeen
beter, aantallen worden uitgedrukt in KVE/gram (Kolonievormende eenheden
per gram silage)), en gebruik een middel waar onderzoek naar gedaan is.
Recent verscheen een interessant samenvattend artikel in \emph{Journal
of Dairy Science}, waarin onderzoek naar voorkomen van pathogene
microben in silage door inkuilmiddelen (Queiroz et al. (2018)).

Een ander recent onderzoek toonde het effect van zuurstofremmende folie
op de kwaliteit van snaimaiskuilen aan. Een kuil werd voor de helft
afgedekt met een normaal polyetyleen afdekkleed
(zuurstofdoorlaatbaarheid was ca 1600 cm3/m2/24 uur) en de andere helft
met een zuurstofremmende folie (zuurstofdoorlaatbaarheid was ca 40
cm3/m2/24 uur). Het deel van de kuil dat met de zuurstofremmende folie
was afgedekt had een lagere pH, en bevatte meer melkzuur, minder
boterzuur, ammoniak, gisten en schimmels dan het deel dat met normale
folie was afdekt. Daarnaast waren de DS in het ene deel ongeveer de
helft van de verliezen in het andere deel (5\% en 10\%). (zie Lima et
al. (2017)).

Een punt van aandacht in verdroogde mais is dat het veel nitraat kan
bevatten. In de VS wordt hier veel voor gewaarschuwd tijdens droge jaren
(zie bijvoorbeeld
\href{https://www.google.com/search?q=drought+stressed+corn+silage}{google
zoektocht}). In het algemeen kan worden gesteld dat verdroogde mais
makkelijk nitraat kan opnemen na een regenbui. Probeer daarom verdroogde
mais voor een regenbui te hakselen. Een deel van het nitraat wordt in de
kuil omgezet in nitriet en vervolgens in stikstofoxides. Sommige
stikstofoxides hebben een roodachtige kleur, andere stikstofoxides zijn
zeer giftig. Stikstofoxides verdwijnen uit de kuil en verdampen na
openen van de kuil zodat er weinig gevaar voor vee bestaat, maar ze zijn
zeer giftig. Let dus goed op bij dit soort kuilen. En, laat kuil ook op
nitraat onderzoeken, hoge nitraatgehaltes zijn giftig voor melkkoeien,
vooral voor drachtige koeien.

\hypertarget{samenvatting}{%
\subsection{Samenvatting}\label{samenvatting}}

\begin{itemize}
\tightlist
\item
  Haksel verdroogde mais als kans op verdere groei afwezig is.
\item
  Gebruik een bewezen inkuilmiddel.
\item
  Kuil zo goed omgelijk in zodat fermentatie zo goed mogelijk verloopt.
\item
  Overweeg gebruik van een zuurstofremmende folie.
\item
  Pas op met hoog nitraatgehalte.
\end{itemize}

\hypertarget{bronnen}{%
\subsection*{Bronnen}\label{bronnen}}
\addcontentsline{toc}{subsection}{Bronnen}

\hypertarget{refs}{}
\leavevmode\hypertarget{ref-Lima2017}{}%
Lima, L.M., J.P. Dos Santos, D.R. Casagrande, C.L.S. Ávila, M.S. Lara,
and T.F. Bernardes. 2017. ``Lining bunker walls with oxygen barrier film
reduces nutrient losses in corn silages.'' \emph{Journal of Dairy
Science}, 1--9. \url{https://doi.org/10.3168/jds.2016-12129}.

\leavevmode\hypertarget{ref-Queiroz2018}{}%
Queiroz, O C M, I M Ogunade, Z Weinberg, and A T Adesogan. 2018.
``Silage review : Foodborne pathogens in silage and their mitigation by
silage additives 1.'' \emph{Journal of Dairy Science} 101 (5). American
Dairy Science Association:4132--42.
\url{https://doi.org/10.3168/jds.2017-13901}.


\end{document}
